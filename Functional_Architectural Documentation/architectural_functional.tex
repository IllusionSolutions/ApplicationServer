\documentclass[paper=a4, fontsize=11pt]{scrartcl} % A4 paper and 11pt font size

\usepackage[english]{babel} % English language/hyphenation
\usepackage{fancyhdr} % Custom headers and footers
\usepackage{graphicx}
\usepackage{xcolor}
\usepackage{titlesec}

\graphicspath{ {images/} }
\pagestyle{fancyplain}

\fancyhead{} 
\fancyfoot[L]{} % Empty left footer
\fancyfoot[C]{} % Empty center footer
\fancyfoot[C]{\thepage} % Page numbering for center footer
\renewcommand{\headrulewidth}{0pt} % Remove header underlines
\renewcommand{\footrulewidth}{0pt} % Remove footer underlines
\setlength{\headheight}{13.6pt} % Customize the height of the header
\titleformat{\section}[block]{\Large\bfseries\filcenter}{}{65pt}{}
\titleformat{\subsection}[hang]{\bfseries}{}{40pt}{}

\title {
	\normalfont \normalsize 
	\textsc{University of Pretoria, Department of Computer Science} \\ [25pt]
	\huge Scope, Vision, Non-Functional \& Functional Requirements Specification\\
}

\author {
	Brandon Wardley  \\
	Mothusi Masibi \\
	Marc Antel \\
	Stuart Andrews \\
}
\date{\normalsize\today} % Today's date or a custom date

\begin{document}
	\maketitle % Print the title
	\newpage
	\section{Scope}
	\newpage
	\section{Vision}
	\newpage
	\section{Initial Architecture Design}
	\newpage
	\section{Functional Requirements}
	\newpage
	\section{Non-functional Requirements}
	\subsection{Quality requirements}
	\paragraph{}
	\subsection{Access channel}
	\newpage
	\section{Proposed Frameworks and Technologies}
	\subsection{Current transformer data transfers}
	Serial Peripheral Interface (SPI) will be used to send data to the photon. SPI is used as a synchronous data bus,
	and uses seperate lines for data and a clock which keeps both sides of the SPI in sync. The receiver (particle photon) detects an incoming 
	edge and will immediately data line. 
	\subsection{Particle Photon}
	Particle API js will be used for communications and transfers from the Particle Photon to the server. The library contains all
	methods that will be needed to connect the device to wifi and post data to the server.
	
	There are no alternatives to this technology.
	\subsection{Data format}
	The data format that will be used is JSON. JSON is a lightweight data-interchange format which is also easy to understand and
	manipulate. The particle already has built in functionality in posting data in JSON format, making it a simpler process.
	
	Alternatively MessagePack could be used. Since MessagePack uses binary serializaton to make faster and smaller data compared to JSON.
	If data size is a concern then MessagePack will be a good alternative.
\end{document}
